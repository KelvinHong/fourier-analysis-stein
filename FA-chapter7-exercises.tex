%%%%%%%%%%%%%%%%%%%%%%%%%%%%%%%%%%%%%%%%%
% Lachaise Assignment
% LaTeX Template
% Version 1.0 (26/6/2018)
%
% This template originates from:
% http://www.LaTeXTemplates.com
%
% Authors:
% Marion Lachaise & François Févotte
% Vel (vel@LaTeXTemplates.com)
%
% License:
% CC BY-NC-SA 3.0 (http://creativecommons.org/licenses/by-nc-sa/3.0/)
% 
%%%%%%%%%%%%%%%%%%%%%%%%%%%%%%%%%%%%%%%%%

%----------------------------------------------------------------------------------------
%	PACKAGES AND OTHER DOCUMENT CONFIGURATIONS
%----------------------------------------------------------------------------------------

\documentclass{article}

\input{structure.tex} % Include the file specifying the document structure and custom commands

%----------------------------------------------------------------------------------------
%	ASSIGNMENT INFORMATION
%----------------------------------------------------------------------------------------

\title{Fourier Analysis Stein: Chapter 7. Problems.} % Title of the assignment

\author{Kelvin Hong\\ \texttt{kh.boon2@gmail.com}} % Author name and email address

\date{Xiamen University Malaysia, Asia Pacific University Malaysia --- \today} % University, school and/or department name(s) and a date

%----------------------------------------------------------------------------------------

\begin{document}

\maketitle % Print the title

\section{Problems}

\begin{enumerate}
    \item Let $f$ be a function on the circle. For each $N\geq 1$ the discrete Fourier coefficients of $f$ are defined by
    $$a_N(n) = \dfrac1N \sum_{k=1}^N f(e^{2\pi ik/N}) e^{-2\pi ikn/N}, \qquad \text{ for } n\in\mathbb Z.$$
    We also let 
    $$a(n) = \int_0^1 f(e^{2\pi ix}) e^{-2\pi nx} dx$$
    denote the ordinary Fourier coefficients of $f$. 
    \begin{enumerate}[(a)]
        \item Show that $a_N(n) = a_N(n+N)$.
        
        \begin{solution}
            The only term related to $n$ is $e^{-2\pi ikn/N}$, so it is trivial to see this term is periodic with a period of $N$.
        \end{solution}

        \item Prove that if $f$ is continuous, then $a_N(n)\to a(n)$ as $N\to\infty$.
        
        \begin{solution}
            Let $n$ be a fixed integer. Since $f$ is continuous and periodic, it is bounded, we assume it is bounded above by a constant $M>0$.
            Given $\varepsilon>0$, we may choose a large enough $N$ such that 
            $$|e^{-2\pi inx} - e^{-2\pi iny}| < \dfrac\varepsilon{2M} \text{ whenever } |x-y|<\dfrac1N.$$
            Also, $f$ is uniformly continuous because it is periodic, so we may choose another large enough $N'$ so that
            $$|f(e^{2\pi i x}) - f(e^{2\pi i y})| < \dfrac\varepsilon2 \text{ whenever } |x-y|<\dfrac1{N'}.$$
            We now simply choose $N_0=\max(N, N')$ so the above two can be satisfied.

            For $N>N_0$ we have
            \begin{align*}
                |a_N(n) - a(n)| &= \abs{
                        \sum_{k=1}^N \left[\dfrac1N f(e^{2\pi ik/N}) e^{-2\pi ikn/N} - 
                        \int_{(k-1)/N}^{k/N} f(e^{2\pi ix}) e^{-2\pi inx}dx\right]
                    } \\
                    &\leq \sum_{k=1}^N \int_{(k-1)/N}^{k/N} \abs{
                        f(e^{2\pi ik/N}) e^{-2\pi ikn/N} - f(e^{2\pi ix})e^{-2\pi inx}
                    }dx.
            \end{align*}

            The inner term can be estimated like so:
            \begin{align*}
                &\abs{
                    f(e^{2\pi ik/N}) e^{-2\pi ikn/N} - f(e^{2\pi ix})e^{-2\pi inx}
                } \\
                \leq &\abs{
                    f(e^{2\pi ik/N}) - f(e^{2\pi ix})
                } + \abs{ f(e^{2\pi ix}) } \abs{
                    e^{-2\pi ikn/N} - e^{-2\pi inx}
                }\\
                < &\varepsilon.
            \end{align*}
            This proves $\lim_{N\to\infty}a_N(n)=a(n)$.
        \end{solution}
    \end{enumerate}
\end{enumerate}



\end{document}

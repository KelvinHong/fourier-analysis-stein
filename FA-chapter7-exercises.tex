%%%%%%%%%%%%%%%%%%%%%%%%%%%%%%%%%%%%%%%%%
% Lachaise Assignment
% LaTeX Template
% Version 1.0 (26/6/2018)
%
% This template originates from:
% http://www.LaTeXTemplates.com
%
% Authors:
% Marion Lachaise & François Févotte
% Vel (vel@LaTeXTemplates.com)
%
% License:
% CC BY-NC-SA 3.0 (http://creativecommons.org/licenses/by-nc-sa/3.0/)
% 
%%%%%%%%%%%%%%%%%%%%%%%%%%%%%%%%%%%%%%%%%

%----------------------------------------------------------------------------------------
%	PACKAGES AND OTHER DOCUMENT CONFIGURATIONS
%----------------------------------------------------------------------------------------

\documentclass{article}

%%%%%%%%%%%%%%%%%%%%%%%%%%%%%%%%%%%%%%%%%
% Lachaise Assignment
% Structure Specification File
% Version 1.0 (26/6/2018)
%
% This template originates from:
% http://www.LaTeXTemplates.com
%
% Authors:
% Marion Lachaise & François Févotte
% Vel (vel@LaTeXTemplates.com)
%
% License:
% CC BY-NC-SA 3.0 (http://creativecommons.org/licenses/by-nc-sa/3.0/)
% 
%%%%%%%%%%%%%%%%%%%%%%%%%%%%%%%%%%%%%%%%%

%----------------------------------------------------------------------------------------
%	PACKAGES AND OTHER DOCUMENT CONFIGURATIONS
%----------------------------------------------------------------------------------------

\usepackage{amsmath,amsfonts,stmaryrd,amssymb,amsthm} % Math packages

\usepackage{enumerate} % Custom item numbers for enumerations

\usepackage{hyperref} % Use href

\usepackage{mathtools} % Use PairedDelimiter

\usepackage[ruled]{algorithm2e} % Algorithms

\usepackage[framemethod=tikz]{mdframed} % Allows defining custom boxed/framed environments

\usepackage{listings} % File listings, with syntax highlighting
\lstset{
	basicstyle=\ttfamily, % Typeset listings in monospace font
}

\usepackage[skip=10pt]{parskip} % Paragraph spacing

%----------------------------------------------------------------------------------------
%	DOCUMENT MARGINS
%----------------------------------------------------------------------------------------

\usepackage{geometry} % Required for adjusting page dimensions and margins

\geometry{
	paper=a4paper, % Paper size, change to letterpaper for US letter size
	top=2.5cm, % Top margin
	bottom=3cm, % Bottom margin
	left=2.5cm, % Left margin
	right=2.5cm, % Right margin
	headheight=14pt, % Header height
	footskip=1.5cm, % Space from the bottom margin to the baseline of the footer
	headsep=1.2cm, % Space from the top margin to the baseline of the header
	%showframe, % Uncomment to show how the type block is set on the page
}

%----------------------------------------------------------------------------------------
%	FONTS
%----------------------------------------------------------------------------------------

\usepackage[utf8]{inputenc} % Required for inputting international characters
\usepackage[T1]{fontenc} % Output font encoding for international characters

\usepackage{XCharter} % Use the XCharter fonts

%----------------------------------------------------------------------------------------
%	COMMAND LINE ENVIRONMENT
%----------------------------------------------------------------------------------------

% Usage:
% \begin{commandline}
%	\begin{verbatim}
%		$ ls
%		
%		Applications	Desktop	...
%	\end{verbatim}
% \end{commandline}

\mdfdefinestyle{commandline}{
	leftmargin=10pt,
	rightmargin=10pt,
	innerleftmargin=15pt,
	middlelinecolor=black!50!white,
	middlelinewidth=2pt,
	frametitlerule=false,
	backgroundcolor=black!5!white,
	frametitle={Command Line},
	frametitlefont={\normalfont\sffamily\color{white}\hspace{-1em}},
	frametitlebackgroundcolor=black!50!white,
	nobreak,
}

% Define a custom environment for command-line snapshots
\newenvironment{commandline}{
	\medskip
	\begin{mdframed}[style=commandline]
}{
	\end{mdframed}
	\medskip
}

%----------------------------------------------------------------------------------------
%	FILE CONTENTS ENVIRONMENT
%----------------------------------------------------------------------------------------

% Usage:
% \begin{file}[optional filename, defaults to "File"]
%	File contents, for example, with a listings environment
% \end{file}

\mdfdefinestyle{file}{
	innertopmargin=1.6\baselineskip,
	innerbottommargin=0.8\baselineskip,
	topline=false, bottomline=false,
	leftline=false, rightline=false,
	leftmargin=2cm,
	rightmargin=2cm,
	singleextra={%
		\draw[fill=black!10!white](P)++(0,-1.2em)rectangle(P-|O);
		\node[anchor=north west]
		at(P-|O){\ttfamily\mdfilename};
		%
		\def\l{3em}
		\draw(O-|P)++(-\l,0)--++(\l,\l)--(P)--(P-|O)--(O)--cycle;
		\draw(O-|P)++(-\l,0)--++(0,\l)--++(\l,0);
	},
	nobreak,
}

% Define a custom environment for file contents
\newenvironment{file}[1][File]{ % Set the default filename to "File"
	\medskip
	\newcommand{\mdfilename}{#1}
	\begin{mdframed}[style=file]
}{
	\end{mdframed}
	\medskip
}

%----------------------------------------------------------------------------------------
%	NUMBERED QUESTIONS ENVIRONMENT
%----------------------------------------------------------------------------------------

% Usage:
% \begin{question}[optional title]
%	Question contents
% \end{question}

\mdfdefinestyle{question}{
	innertopmargin=1.2\baselineskip,
	innerbottommargin=0.8\baselineskip,
	roundcorner=5pt,
	nobreak,
	singleextra={%
		\draw(P-|O)node[xshift=1em,anchor=west,fill=white,draw,rounded corners=5pt]{%
		Question \theQuestion\questionTitle};
	},
}

\newcounter{Question} % Stores the current question number that gets iterated with each new question

% Define a custom environment for numbered questions
\newenvironment{question}[1][\unskip]{
	\bigskip
	\stepcounter{Question}
	\newcommand{\questionTitle}{~#1}
	\begin{mdframed}[style=question]
}{
	\end{mdframed}
	\medskip
}

%----------------------------------------------------------------------------------------
%	WARNING TEXT ENVIRONMENT
%----------------------------------------------------------------------------------------

% Usage:
% \begin{warn}[optional title, defaults to "Warning:"]
%	Contents
% \end{warn}

\mdfdefinestyle{warning}{
	topline=false, bottomline=false,
	leftline=false, rightline=false,
	nobreak,
	singleextra={%
		\draw(P-|O)++(-0.5em,0)node(tmp1){};
		\draw(P-|O)++(0.5em,0)node(tmp2){};
		\fill[black,rotate around={45:(P-|O)}](tmp1)rectangle(tmp2);
		\node at(P-|O){\color{white}\scriptsize\bf !};
		\draw[very thick](P-|O)++(0,-1em)--(O);%--(O-|P);
	}
}

% Define a custom environment for warning text
\newenvironment{warn}[1][Warning:]{ % Set the default warning to "Warning:"
	\medskip
	\begin{mdframed}[style=warning]
		\noindent{\textbf{#1}}
}{
	\end{mdframed}
}

%----------------------------------------------------------------------------------------
%	INFORMATION ENVIRONMENT
%----------------------------------------------------------------------------------------

% Usage:
% \begin{info}[optional title, defaults to "Info:"]
% 	contents
% 	\end{info}

\mdfdefinestyle{info}{%
	topline=false, bottomline=false,
	leftline=false, rightline=false,
	nobreak,
	singleextra={%
		\fill[black](P-|O)circle[radius=0.4em];
		\node at(P-|O){\color{white}\scriptsize\bf i};
		\draw[very thick](P-|O)++(0,-0.8em)--(O);%--(O-|P);
	}
}

% Define a custom environment for information
\newenvironment{info}[1][Info:]{ % Set the default title to "Info:"
	\medskip
	\begin{mdframed}[style=info]
		\noindent{\textbf{#1}}
}{
	\end{mdframed}
}

% Kelvin Hong: Below are my custom solution environment.
\newenvironment{solution}
{ % Before writing the solution
    \textcolor{blue}{\textbf{Solution: }}
}
{ % At the end of the solution.
    \hfill\textcolor{green}{\qed}
}

\newenvironment{theorem}
{ % Before writing the solution
    \textcolor{blue}{\textbf{Theorem }}
}
{ % At the end of the solution.
    \hfill\textcolor{green}{\textit{theoremend.}}
}

% Create a partial differentiation symbol. 
%% display-style partial differentiation.
\newcommand{\dpard}[3][]{%
    \dfrac{\partial^{#1} #2}{\partial #3^{#1}}%
}
%% inline-style partial differentiation.
\newcommand{\pard}[3][]{%
    \frac{\partial^{#1} #2}{\partial #3^{#1}}%
}

% Below taken from https://tex.stackexchange.com/questions/43008/absolute-value-symbols
\DeclarePairedDelimiter\abs{\lvert}{\rvert}%
\DeclarePairedDelimiter\norm{\lVert}{\rVert}%

% Swap the definition of \abs* and \norm*, so that \abs
% and \norm resizes the size of the brackets, and the 
% starred version does not.
\makeatletter
\let\oldabs\abs
\def\abs{\@ifstar{\oldabs}{\oldabs*}}
%
\let\oldnorm\norm
\def\norm{\@ifstar{\oldnorm}{\oldnorm*}}
\makeatother

% Use slanted v
\DeclareSymbolFont{matha}{OML}{txmi}{m}{it}% txfonts
\DeclareMathSymbol{\varv}{\mathord}{matha}{118}

 % Include the file specifying the document structure and custom commands

%----------------------------------------------------------------------------------------
%	ASSIGNMENT INFORMATION
%----------------------------------------------------------------------------------------

\title{Fourier Analysis Stein: Chapter 7. Problems.} % Title of the assignment

\author{Kelvin Hong\\ \texttt{kh.boon2@gmail.com}} % Author name and email address

\date{Xiamen University Malaysia, Asia Pacific University Malaysia --- \today} % University, school and/or department name(s) and a date

%----------------------------------------------------------------------------------------

\begin{document}

\maketitle % Print the title

\section{Problems}

\begin{enumerate}
    \item Let $f$ be a function on the circle. For each $N\geq 1$ the discrete Fourier coefficients of $f$ are defined by
    $$a_N(n) = \dfrac1N \sum_{k=1}^N f(e^{2\pi ik/N}) e^{-2\pi ikn/N}, \qquad \text{ for } n\in\mathbb Z.$$
    We also let 
    $$a(n) = \int_0^1 f(e^{2\pi ix}) e^{-2\pi nx} dx$$
    denote the ordinary Fourier coefficients of $f$. 
    \begin{enumerate}[(a)]
        \item Show that $a_N(n) = a_N(n+N)$.
        
        \begin{solution}
            The only term related to $n$ is $e^{-2\pi ikn/N}$, so it is trivial to see this term is periodic with a period of $N$.
        \end{solution}

        \item Prove that if $f$ is continuous, then $a_N(n)\to a(n)$ as $N\to\infty$.
        
        \begin{solution}
            Let $n$ be a fixed integer. Since $f$ is continuous and periodic, it is bounded, we assume it is bounded above by a constant $M>0$.
            Given $\varepsilon>0$, we may choose a large enough $N$ such that 
            $$|e^{-2\pi inx} - e^{-2\pi iny}| < \dfrac\varepsilon{2M} \text{ whenever } |x-y|<\dfrac1N.$$
            Also, $f$ is uniformly continuous because it is periodic, so we may choose another large enough $N'$ so that
            $$|f(e^{2\pi i x}) - f(e^{2\pi i y})| < \dfrac\varepsilon2 \text{ whenever } |x-y|<\dfrac1{N'}.$$
            We now simply choose $N_0=\max(N, N')$ so the above two can be satisfied.

            For $N>N_0$ we have
            \begin{align*}
                |a_N(n) - a(n)| &= \abs{
                        \sum_{k=1}^N \left[\dfrac1N f(e^{2\pi ik/N}) e^{-2\pi ikn/N} - 
                        \int_{(k-1)/N}^{k/N} f(e^{2\pi ix}) e^{-2\pi inx}dx\right]
                    } \\
                    &\leq \sum_{k=1}^N \int_{(k-1)/N}^{k/N} \abs{
                        f(e^{2\pi ik/N}) e^{-2\pi ikn/N} - f(e^{2\pi ix})e^{-2\pi inx}
                    }dx.
            \end{align*}

            The inner term can be estimated like so:
            \begin{align*}
                &\abs{
                    f(e^{2\pi ik/N}) e^{-2\pi ikn/N} - f(e^{2\pi ix})e^{-2\pi inx}
                } \\
                \leq &\abs{
                    f(e^{2\pi ik/N}) - f(e^{2\pi ix})
                } + \abs{ f(e^{2\pi ix}) } \abs{
                    e^{-2\pi ikn/N} - e^{-2\pi inx}
                }\\
                < &\varepsilon.
            \end{align*}
            This proves $\lim_{N\to\infty}a_N(n)=a(n)$.
        \end{solution}
    \end{enumerate}
    
    \item If $f$ is a $C^1$ function on the circle, prove that $\abs{a_N(n)} \leq c/\abs{n}$ whenever $0<\abs{n}\leq N/2$.

    \begin{solution}
        We have to clarify that $c$ will be a constant that is independent from $n$ and $N$, possible related only to $f$.
        We also only prove the statement for $0<n\leq N/2$, i.e., positive $n$, the proof for negative $n$ should be nearly identical.

        Since $f$ is periodic and $C^1$, it is Lipchitz continuous. 
        \href{https://math.stackexchange.com/questions/4353734/if-f-is-periodic-and-c1-class-then-lipschitz-continuous}{See this answer.}
        Therefore, we let $M$ be a positive constant such that
        $$\abs{f(x)-f(y)} \leq M \abs{x-y} \quad \text{ for all }x,y \text{ on the circle.}$$

        We also use the identity: $|1-e^{i\theta}|=2\sin(\theta/2)\leq \theta$ for $\theta\in[0,\pi]$.

        Following the hint from the book, we let $\ell=1$ if $N/4< n\leq N/2$. If otherwise $n\leq N/4$, we let $\ell$ be the
        largest integer satisfying $\ell n/N\leq 1/2$, then since $(\ell+1)n/N>1/2$, we have 
        $$\abs{\frac12 - \dfrac{\ell n}{N}} = \frac12 - \dfrac{\ell n}{N} < \dfrac nN \leq \frac14.$$
        This is to verify that we can always choose $\ell$ such that $1/4 \leq \ell n/N \leq 1/2$ by construction, and that we also have
        the following lower bound:
        $$\dfrac{\pi}{2} \leq \dfrac{2\pi \ell n}{N} \leq \pi \implies \abs{1-e^{2\pi i\ell n/N}}\geq \sqrt2.$$

        Everything is ready, we now have
        \begin{align*}
            \abs{a_N(n)} \sqrt2 &\leq \abs{a_N(n)} \abs{1-e^{2\pi i\ell n/N}}\\
            &\leq \frac1N \sum_{k=1}^N M\abs{1-e^{2\pi i\ell/N}}\\
            &=  M\abs{1-e^{2\pi i\ell/N}}\\
            &\leq M\cdot \dfrac{2\pi \ell}N\\
            \therefore \abs{a_N(n)}&\leq \dfrac{M\pi}{\sqrt2 n}.
        \end{align*}
    \end{solution}
\end{enumerate}



\end{document}

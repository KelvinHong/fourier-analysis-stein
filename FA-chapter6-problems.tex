%%%%%%%%%%%%%%%%%%%%%%%%%%%%%%%%%%%%%%%%%
% Lachaise Assignment
% LaTeX Template
% Version 1.0 (26/6/2018)
%
% This template originates from:
% http://www.LaTeXTemplates.com
%
% Authors:
% Marion Lachaise & François Févotte
% Vel (vel@LaTeXTemplates.com)
%
% License:
% CC BY-NC-SA 3.0 (http://creativecommons.org/licenses/by-nc-sa/3.0/)
% 
%%%%%%%%%%%%%%%%%%%%%%%%%%%%%%%%%%%%%%%%%

%----------------------------------------------------------------------------------------
%	PACKAGES AND OTHER DOCUMENT CONFIGURATIONS
%----------------------------------------------------------------------------------------

\documentclass{article}

%%%%%%%%%%%%%%%%%%%%%%%%%%%%%%%%%%%%%%%%%
% Lachaise Assignment
% Structure Specification File
% Version 1.0 (26/6/2018)
%
% This template originates from:
% http://www.LaTeXTemplates.com
%
% Authors:
% Marion Lachaise & François Févotte
% Vel (vel@LaTeXTemplates.com)
%
% License:
% CC BY-NC-SA 3.0 (http://creativecommons.org/licenses/by-nc-sa/3.0/)
% 
%%%%%%%%%%%%%%%%%%%%%%%%%%%%%%%%%%%%%%%%%

%----------------------------------------------------------------------------------------
%	PACKAGES AND OTHER DOCUMENT CONFIGURATIONS
%----------------------------------------------------------------------------------------

\usepackage{amsmath,amsfonts,stmaryrd,amssymb,amsthm} % Math packages

\usepackage{enumerate} % Custom item numbers for enumerations

\usepackage{hyperref} % Use href

\usepackage{mathtools} % Use PairedDelimiter

\usepackage[ruled]{algorithm2e} % Algorithms

\usepackage[framemethod=tikz]{mdframed} % Allows defining custom boxed/framed environments

\usepackage{listings} % File listings, with syntax highlighting
\lstset{
	basicstyle=\ttfamily, % Typeset listings in monospace font
}

\usepackage[skip=10pt]{parskip} % Paragraph spacing

%----------------------------------------------------------------------------------------
%	DOCUMENT MARGINS
%----------------------------------------------------------------------------------------

\usepackage{geometry} % Required for adjusting page dimensions and margins

\geometry{
	paper=a4paper, % Paper size, change to letterpaper for US letter size
	top=2.5cm, % Top margin
	bottom=3cm, % Bottom margin
	left=2.5cm, % Left margin
	right=2.5cm, % Right margin
	headheight=14pt, % Header height
	footskip=1.5cm, % Space from the bottom margin to the baseline of the footer
	headsep=1.2cm, % Space from the top margin to the baseline of the header
	%showframe, % Uncomment to show how the type block is set on the page
}

%----------------------------------------------------------------------------------------
%	FONTS
%----------------------------------------------------------------------------------------

\usepackage[utf8]{inputenc} % Required for inputting international characters
\usepackage[T1]{fontenc} % Output font encoding for international characters

\usepackage{XCharter} % Use the XCharter fonts

%----------------------------------------------------------------------------------------
%	COMMAND LINE ENVIRONMENT
%----------------------------------------------------------------------------------------

% Usage:
% \begin{commandline}
%	\begin{verbatim}
%		$ ls
%		
%		Applications	Desktop	...
%	\end{verbatim}
% \end{commandline}

\mdfdefinestyle{commandline}{
	leftmargin=10pt,
	rightmargin=10pt,
	innerleftmargin=15pt,
	middlelinecolor=black!50!white,
	middlelinewidth=2pt,
	frametitlerule=false,
	backgroundcolor=black!5!white,
	frametitle={Command Line},
	frametitlefont={\normalfont\sffamily\color{white}\hspace{-1em}},
	frametitlebackgroundcolor=black!50!white,
	nobreak,
}

% Define a custom environment for command-line snapshots
\newenvironment{commandline}{
	\medskip
	\begin{mdframed}[style=commandline]
}{
	\end{mdframed}
	\medskip
}

%----------------------------------------------------------------------------------------
%	FILE CONTENTS ENVIRONMENT
%----------------------------------------------------------------------------------------

% Usage:
% \begin{file}[optional filename, defaults to "File"]
%	File contents, for example, with a listings environment
% \end{file}

\mdfdefinestyle{file}{
	innertopmargin=1.6\baselineskip,
	innerbottommargin=0.8\baselineskip,
	topline=false, bottomline=false,
	leftline=false, rightline=false,
	leftmargin=2cm,
	rightmargin=2cm,
	singleextra={%
		\draw[fill=black!10!white](P)++(0,-1.2em)rectangle(P-|O);
		\node[anchor=north west]
		at(P-|O){\ttfamily\mdfilename};
		%
		\def\l{3em}
		\draw(O-|P)++(-\l,0)--++(\l,\l)--(P)--(P-|O)--(O)--cycle;
		\draw(O-|P)++(-\l,0)--++(0,\l)--++(\l,0);
	},
	nobreak,
}

% Define a custom environment for file contents
\newenvironment{file}[1][File]{ % Set the default filename to "File"
	\medskip
	\newcommand{\mdfilename}{#1}
	\begin{mdframed}[style=file]
}{
	\end{mdframed}
	\medskip
}

%----------------------------------------------------------------------------------------
%	NUMBERED QUESTIONS ENVIRONMENT
%----------------------------------------------------------------------------------------

% Usage:
% \begin{question}[optional title]
%	Question contents
% \end{question}

\mdfdefinestyle{question}{
	innertopmargin=1.2\baselineskip,
	innerbottommargin=0.8\baselineskip,
	roundcorner=5pt,
	nobreak,
	singleextra={%
		\draw(P-|O)node[xshift=1em,anchor=west,fill=white,draw,rounded corners=5pt]{%
		Question \theQuestion\questionTitle};
	},
}

\newcounter{Question} % Stores the current question number that gets iterated with each new question

% Define a custom environment for numbered questions
\newenvironment{question}[1][\unskip]{
	\bigskip
	\stepcounter{Question}
	\newcommand{\questionTitle}{~#1}
	\begin{mdframed}[style=question]
}{
	\end{mdframed}
	\medskip
}

%----------------------------------------------------------------------------------------
%	WARNING TEXT ENVIRONMENT
%----------------------------------------------------------------------------------------

% Usage:
% \begin{warn}[optional title, defaults to "Warning:"]
%	Contents
% \end{warn}

\mdfdefinestyle{warning}{
	topline=false, bottomline=false,
	leftline=false, rightline=false,
	nobreak,
	singleextra={%
		\draw(P-|O)++(-0.5em,0)node(tmp1){};
		\draw(P-|O)++(0.5em,0)node(tmp2){};
		\fill[black,rotate around={45:(P-|O)}](tmp1)rectangle(tmp2);
		\node at(P-|O){\color{white}\scriptsize\bf !};
		\draw[very thick](P-|O)++(0,-1em)--(O);%--(O-|P);
	}
}

% Define a custom environment for warning text
\newenvironment{warn}[1][Warning:]{ % Set the default warning to "Warning:"
	\medskip
	\begin{mdframed}[style=warning]
		\noindent{\textbf{#1}}
}{
	\end{mdframed}
}

%----------------------------------------------------------------------------------------
%	INFORMATION ENVIRONMENT
%----------------------------------------------------------------------------------------

% Usage:
% \begin{info}[optional title, defaults to "Info:"]
% 	contents
% 	\end{info}

\mdfdefinestyle{info}{%
	topline=false, bottomline=false,
	leftline=false, rightline=false,
	nobreak,
	singleextra={%
		\fill[black](P-|O)circle[radius=0.4em];
		\node at(P-|O){\color{white}\scriptsize\bf i};
		\draw[very thick](P-|O)++(0,-0.8em)--(O);%--(O-|P);
	}
}

% Define a custom environment for information
\newenvironment{info}[1][Info:]{ % Set the default title to "Info:"
	\medskip
	\begin{mdframed}[style=info]
		\noindent{\textbf{#1}}
}{
	\end{mdframed}
}

% Kelvin Hong: Below are my custom solution environment.
\newenvironment{solution}
{ % Before writing the solution
    \textcolor{blue}{\textbf{Solution: }}
}
{ % At the end of the solution.
    \hfill\textcolor{green}{\qed}
}

\newenvironment{theorem}
{ % Before writing the solution
    \textcolor{blue}{\textbf{Theorem }}
}
{ % At the end of the solution.
    \hfill\textcolor{green}{\textit{theoremend.}}
}

% Create a partial differentiation symbol. 
%% display-style partial differentiation.
\newcommand{\dpard}[3][]{%
    \dfrac{\partial^{#1} #2}{\partial #3^{#1}}%
}
%% inline-style partial differentiation.
\newcommand{\pard}[3][]{%
    \frac{\partial^{#1} #2}{\partial #3^{#1}}%
}

% Below taken from https://tex.stackexchange.com/questions/43008/absolute-value-symbols
\DeclarePairedDelimiter\abs{\lvert}{\rvert}%
\DeclarePairedDelimiter\norm{\lVert}{\rVert}%

% Swap the definition of \abs* and \norm*, so that \abs
% and \norm resizes the size of the brackets, and the 
% starred version does not.
\makeatletter
\let\oldabs\abs
\def\abs{\@ifstar{\oldabs}{\oldabs*}}
%
\let\oldnorm\norm
\def\norm{\@ifstar{\oldnorm}{\oldnorm*}}
\makeatother

% Use slanted v
\DeclareSymbolFont{matha}{OML}{txmi}{m}{it}% txfonts
\DeclareMathSymbol{\varv}{\mathord}{matha}{118}

 % Include the file specifying the document structure and custom commands

%----------------------------------------------------------------------------------------
%	ASSIGNMENT INFORMATION
%----------------------------------------------------------------------------------------

\title{Fourier Analysis Stein: Chapter 6. Problems.} % Title of the assignment

\author{Kelvin Hong\\ \texttt{kh.boon2@gmail.com}} % Author name and email address

\date{Xiamen University Malaysia, Asia Pacific University Malaysia --- \today} % University, school and/or department name(s) and a date

%----------------------------------------------------------------------------------------

\begin{document}

\maketitle % Print the title

\section{Problems}

\begin{enumerate}
    \item Let $J_n$ denote the $n^{th}$ order Bessel function, for $n\in \mathbb Z$. Prove that 
    \begin{enumerate}[(a)]
        \item $J_n(\rho)$ is real for all real $\rho$. 

        \begin{solution}
            Taking the complex conjugate of the Bessel function, we have
            \begin{align*}
                2\pi \overline{J_n(\rho)} &= 
                \int_0^{2\pi} e^{-i\rho\sin\theta} e^{in\theta}\, d\theta \\
                &= \int_0^{-2\pi} e^{i\rho\sin\theta} e^{-in\theta}\, (-d\theta) \\
                &= \int_{-2\pi}^0 e^{i\rho\sin\theta} e^{-in\theta}\, d\theta \\
                &= \int_0^{2\pi} e^{i\rho\sin\theta} e^{-in\theta}\, d\theta \text{ (Periodicity) } \\ 
                &= 2\pi J_n(\rho).
            \end{align*}
            This shows $J_n(\rho)$ is real when $\rho$ is real.
        \end{solution}
        

        \item $J_{-n}(\rho)=(-1)^n J_n(\rho)$.

        \begin{solution}
            The question is equivalent of showing $J_{-n}(\rho)+J_n(\rho)=0$ when $n$ is odd and 
            $J_{-n}(\rho)-J_n(\rho)=0$ when $n$ is even. 

            When $n$ is even, we have 
            \begin{align*}
                J_n(\rho)-J_{-n}(\rho) &= \frac{i}{\pi} \int_0^{2\pi} e^{i\rho\sin\theta} \sin(n\theta)\, d\theta
            \end{align*}

            The second half of the integral can be transformed into below by substitution $u=2\pi-\theta$:
            \begin{align*}
                \int_\pi^{2\pi} e^{i\rho\sin\theta} \sin(n\theta)\, d\theta 
                &= \int_\pi^0 e^{i\rho\sin(2\pi-u)} \sin(n(2\pi-u))\, (-du) \\
                &= -\int_0^\pi e^{-i\rho\sin u} \sin(nu)\, du \\
            \end{align*}
            Notice this is true for all integer $n$, not just even $n$, but we will use it in the following derivation 
            with another substitution $t=\theta-\pi/2$:
            \begin{align*}
                J_n(\rho)-J_{-n}(\rho) &= \frac{2i}{\pi} \int_0^{\pi} \cos(\rho\sin\theta) \sin(n\theta)\, d\theta \\
                &= \frac{2i}{\pi} \int_{-\pi/2}^{\pi/2} \cos(\rho\sin(t+\pi/2)) \sin(n(t+\pi/2))\, dt \\
                &= \pm \frac{2i}{\pi} \int_{-\pi/2} ^ {\pi/2} \cos(\rho \cos t) \sin(nt)\, dt \\
                &= 0
            \end{align*}
            On the last second line, when $n\equiv 0\pmod 4$ it will takes the positive sign, and when $n\equiv 2\pmod 4$ 
            it will takes the negative sign, but either way the integrand is an odd function on $t$ and the value will be zero. 

            A similar two-times variable substitution when $n$ is odd will give us the proof for $J_{-n}(\rho)+J_n(\rho)=0$.
        \end{solution}

        \item $2J_n'(\rho)=J_{n-1}(\rho)-J_{n+1}(\rho)$.
        
        \begin{solution}
            We have
            \begin{align*}
                2J_n'(\rho) &= \frac1{\pi} \int_0^{2\pi} ie^{i\rho\sin\theta} e^{-in\theta}\sin\theta \, d\theta \\
                &= \frac1\pi \int_0^{2\pi} e^{i\rho\sin\theta} e^{-in\theta}\dfrac{e^{i\theta}-e^{-i\theta}}2 \, d\theta \\
                &= J_{n-1}(\rho)-J_{n+1}(\rho).
            \end{align*}
        \end{solution}

        \item $\left(\frac{2n}\rho\right) J_n(\rho) = J_{n-1}(\rho) + J_{n+1}(\rho)$.
        
        \begin{solution}
            This identity can be proven using integration by parts. RHS is 
            $$\frac1{2\pi} \int_0^{2\pi} e^{i\rho\sin\theta} e^{-in\theta} \cdot 2\cos\theta \, d\theta.$$

            Consider letting $u=e^{-in\theta}$, $dv=2e^{i\rho\sin\theta} \cos\theta d\theta$, then we have 
            $du =-in e^{-in\theta} d\theta$, $v = \frac2{i\rho} e^{i\rho \sin\theta}. $

            Therefore, 
            \begin{align*}
                \text{RHS} &= \frac1{2\pi} \left(\left[\frac2{i\rho}e^{i\rho\sin\theta} e^{-in\theta}\right]_0^{2\pi} 
                    + \frac{2n}\rho\int_0^{2\pi} e^{i\rho\sin\theta}e^{-in\theta}\, d\theta\right) \\
                    &= \frac1{2\pi} \left(\frac2{i\rho} - \frac2{i\rho}\right) + \frac{2n}\rho J_n(\rho) \\
                    &= \frac{2n}\rho J_n(\rho).
            \end{align*}
        \end{solution}

        \item $(\rho^{-n}J_n(\rho))' = -\rho^{-n}J_{n+1}(\rho)$. 
        
        \begin{solution}
            It will be easier to prove this identity by using previous identities. We have
            \begin{align*}
                LHS &= -n\rho^{-n-1}J_n(\rho) + \rho^{-n}J_n(\rho)' \\
                &= -\dfrac{\rho^{-n}}{2} \cdot \left(\dfrac{2n}\rho J_n(\rho)\right) + \dfrac{\rho^{-n}}2(J_{n-1}(\rho) - J_{n+1}(\rho)) \\
                &= -\dfrac{\rho^{-n}}2(J_{n-1}(\rho) + J_{n+1}(\rho)) + \dfrac{\rho^{-n}}2(J_{n-1}(\rho) - J_{n+1}(\rho)) \\
                &= -\rho^{-n} J_{n+1}(\rho).
            \end{align*}
        \end{solution}

        \item $(\rho^n J_n(\rho))' = \rho^n J_{n-1}(\rho)$.
        
        \begin{solution}
            Similar as previous identity, we have 

            \begin{align*}
                LHS &= n\rho^{n-1} J_n(\rho) + \rho^n J_n'(\rho) \\
                &= \frac{\rho^n}2(J_{n-1}(\rho) + J_{n+1}(\rho)) + \frac{\rho^n}2(J_{n-1}(\rho) - J_{n+1}(\rho)) \\
                &= \rho^n J_{n-1}(\rho).
            \end{align*}
        \end{solution}

        \item $J_n(\rho)$ satisfies the second order differential equation
        $$J_n''(\rho) + \rho^{-1}J_n'(\rho) + (1-n^2/\rho^2)J_n(\rho) = 0.$$

        \begin{solution}
            We basically need to use the identities from (e) and (f), noticing how they can decrease the index $n\to n-1$ and then 
            increase it again $n\to n+1$. 

            We use $J_n$ as a shorthand notation for $J_n(\rho)$ temporarily for the derivation below.
            \begin{align*}
                \because (\rho^n J_n)' &= \rho^n J_{n-1} \\
                n\rho^{n-1} J_n + \rho^n J_n' &= \rho^n J_{n-1} \\
                n\rho^{-n}J_n + \rho^{-n+1}J_n' &= \rho^{-(n-1)}J_{n-1} \\
                \therefore \rho^{-n+1}J_n''-(n-1)\rho^{-n}J_n'+n\rho^{-n}J_n'-n^2\rho^{-n-1}J_n&=(\rho^{-(n-1)} J_{n-1})'\\
                \rho^{-n+1}J_n'' + \rho^{-n}J_n' - n^2\rho^{-n-1} J_n &= -\rho^{-n+1}J_n\\
                J_n'' + \rho^{-1}J_n' + (1-n^2\rho^{-2}) J_n&=0.
            \end{align*}
        \end{solution}

        \item Show that 
        
        \begin{solution}
            $$J_n(\rho) = \left(\dfrac\rho2\right)^n \sum_{m=0}^\infty (-1)^m \dfrac{\rho^{2m}}{2^{2m}m!(n+m)!}.$$

            Notice that from (g), $J_n(\rho)$ satisfies a second order ODE in the form of $\rho^2 J_n'' + \rho J_n' + (\rho^2-n^2)J_n=0$, 
            which can be solved by Frobenius method. 

            For some convenient reason, let's assume $n\geq 1$ for now. We will then prove for $n=0$. Since there is factorial, we assume 
            we only have to prove for $n\geq 0$. 

            Let $J_n(\rho)$ has the form
            $$J_n(\rho) = \sum_{k=0}^\infty A_k\rho^{k+r}, \quad A_0\neq 0.$$
            Then we have 
            \begin{align*}
                J_n'(\rho) &= \sum_{k=0}^\infty (k+r)A_k \rho^{k+r-1}, \\
                J_n''(\rho) &= \sum_{k=0}^\infty (k+r)(k+r-1) A_k \rho^{k+r-2}.
            \end{align*}

            Substitutes into the second order ODE, we have 
            $$\sum_{k=0}^\infty [(k+r)(k+r-1) + k+r+\rho^2-n^2]A_k\rho^{k+r} = 0.$$

            Cancel out $\rho^r$, using the assumption that $A_0\neq 0$ and substitute $\rho=0$, we obtain the indicial equation: 
            $$r(r-1)+r-n^2=0 \implies r=\pm n.$$

            Seems like we have two possibilities for $r$, but $r=-n$ is actually impossible because $J_n(\rho)$ is a finite function, 
            it wouldn't have any singularity. 
            We thus have $r=n$. 

            Simplify the equations, we have 
            \begin{align*}
                \sum_{k=0}^\infty [(k+n)(k+n-1) + k+n+\rho^2-n^2] A_k \rho^k &= 0\\
                \sum_{k=0}^\infty (k^2+2kn+\rho^2) A_k\rho^k&=0 \\
                \sum_{k=0}^\infty (k^2+2kn)A_k\rho^k + \sum_{k=0}^\infty A_k\rho^{k+2} &= 0\\
                (2n+1)A_1 \rho + \sum_{k=2}^\infty [k(k+2n) A_k + A_{k-2}] \rho^k&=0.
            \end{align*}

            The first term tells us $A_1=0$ since we can cancel a $\rho$ everywhere and let $\rho=0$.
            Then we also have $A_k = -\frac1{k(k+2n)} A_{k-2} $ for $k\geq 2$. 
            This shows that $A_{2m+1}=0$ for all integer $m\geq 0$, and we made some progress that proves the series only has 
            even power terms.
            
            Now let $k=2m$ for $m\geq 1$, we have 
            \begin{align*}
                A_{2m} &= -\frac1{4m(m+n)} A_{2m-2}  \\
                &= (-1)^2 \dfrac{A^{2m-4}}{4^2 m(m-1)(m+n)(m-1+n)} \\
                &= (-1)^m \dfrac{A_0 n!}{4^mm!(m+n)!}.
            \end{align*}

            We now only have to show $A_0=1/(2^n n!)$ and the proof will be done for $n\geq 1$.
            To show this, we first notice that as $r=n$, that $\lim_{\rho\to 0}\rho^{-m}J_n(\rho) = 0$ for any integer $m<n$.  
            Then, we prove by induction on $n$: We denote temporarily $A_0(n)$ to be our $A_0$ coefficient and want to show that 
            $A_0(n) = \frac1{2^nn!}$. 

            First it is not hard to see $A_0(n) = \lim_{\rho\to 0} \rho^{-n}J_n(\rho)$, for $n=0$ we have 
            $J_0(0)=\frac1{2\pi} \int_0^{2\pi} d\theta=1$, and for $n=1$ we use the identity from (d): 
            \begin{align*}
                \frac{J_1(\rho)}{\rho} &= \frac{J_0(\rho) + J_2(\rho)}{2} \\
                \therefore A_0(1) &=\lim_{\rho\to 0} \frac{J_1(\rho)}{\rho} \\
                &= \frac12.
            \end{align*}

            Now assumes $A_0(n) = \frac1{2^nn!}$ for some $n\geq 1$, then 
            \begin{align*}
                \frac{J_{n+1}(\rho)}{\rho} &= \frac{J_n(\rho) + J_{n+2}(\rho)}{2n+2} \\
                \frac{J_{n+1}(\rho)}{\rho^{n+1}} &= \frac{\rho^{-n}J_n(\rho) + \rho^{-n} J_{n+2}(\rho)}{2(n+1)} \\
                \therefore A_0(n+1) &= \dfrac{A_0(n)}{2(n+1)} \\
                &= \frac{1}{2^{n+1}(n+1)!}.
            \end{align*}

            This shows the series expansion for $J_n(\rho)$ is true for $n\geq 1$.  

            To prove the series expansion is also true when $n=0$, we just use the identity from (f) when $n=1$. 
        \end{solution}

        \item Show that for all integers $n$ and all real numbers $a$ and $b$ we have 
        $$J_n(a+b) = \sum_{\ell\in\mathbb Z}J_\ell(a) J_{n-\ell}(b).$$

        \begin{solution}
            By definition of Bessel's function, the Fourier series of $e^{i\rho\sin\theta}$ is 
            $$e^{i\rho\sin\theta} = \sum_{n\in\mathbb Z} J_n(\rho) e^{in\theta}.$$

            Using this, we can derive the identity.

            \begin{align*}
                J_n(a+b) &= \frac1{2\pi} \int_0^{2\pi} e^{i(a+b)\sin\theta} e^{-in\theta} d\theta \\
                &= \frac1{2\pi} \int_0^{2\pi} \left(\sum_{\ell\in\mathbb Z} J_\ell (a) e^{i\ell\theta}\right)
                    \left(\sum_{m\in\mathbb Z} J_m (a) e^{im\theta}\right) e^{-in\theta} d\theta\\
                    &= \sum_{\ell\in\mathbb Z}\sum_{m\in\mathbb Z} J_\ell(a) J_m(b) \frac1{2\pi} \int_0^{2\pi} e^{i(\ell+m-n)\theta} d\theta\\
                    &= \sum_{\ell\in\mathbb Z} J_{\ell}(a) J_{n-\ell}(b). 
            \end{align*}
            
            The last equation is because $\int_0^{2\pi} e^{ik\theta} d\theta=2\pi$ when $k=0$, but become zero when $k$ is any non-zero integer.
        \end{solution}
        
    \end{enumerate}

    \item Another formula for $J_n(\rho)$ that allows one to define Bessel functions for non-integral values of $n$, $(n > -1/2)$ is 
    $$J_n(\rho) = \dfrac{(\rho/2)^n}{\Gamma(n+1/2)\sqrt\pi} \int_{-1}^1 e^{i\rho t} (1-t^2)^{n-(1/2)}\, dt.$$

    \begin{enumerate}[(a)]
        \item Check that the above formula agrees with the definition of $J_n(\rho)$ for integral $n\geq 0$. 
        
        \begin{solution}
            When $n=0$, the RHS become
            \begin{align*}
                RHS&= \dfrac1\pi \int_{-1}^1 \dfrac{e^{i\rho t}}{\sqrt{1-t^2}} \, dt.
            \end{align*}
            Let $t=\sin\theta$, we have
            $$RHS = \dfrac1\pi \int_{-\pi/2}^{\pi/2} e^{i\rho \sin\theta} \, d\theta 
            = \dfrac1{2\pi} \int_0^{2\pi} e^{i\rho \sin\theta} \, d\theta = J_0(\rho). $$
            
            Now assume that the identify works for some non-negative integer $n$, using the 
            identity from 1(e) and integration by parts, we see that 
            \begin{align*}
                J_{n+1}(\rho) &= -\rho^n \dfrac{d}{d\rho} (\rho^{-n} J_n(\rho))\\
                &= -\rho^n \dfrac{d}{d\rho} \left[\dfrac1{2^n \Gamma(n+1/2)\sqrt\pi} 
                    \int_{-1}^1 e^{i\rho t}(1-t^2)^{n-1/2}dt\right] \\
                &= -\dfrac{i\rho^n}{2^n\Gamma(n+1/2)\sqrt\pi} \int_{-1}^1 e^{i\rho t} t(1-t^2)^{n-1/2}dt \\
                &= -\dfrac{i\rho^n}{2^n\Gamma(n+1/2)\sqrt\pi} \left[-\dfrac{e^{i\rho t}(1-t^2)^{n+1/2}}{2n+1} \bigg|_{-1}^1
                    +\dfrac{i\rho}{2(n+1/2)} \int_{-1}^1 e^{i\rho t} (1-t^2)^{n+1/2}dt\right]\\
                &= \left(\dfrac\rho2\right)^{n+1} \dfrac1{\Gamma(n+3/2)\sqrt\pi} \int_{-1}^1 e^{i\rho t}(1-t^2)^{n+1/2}dt.
            \end{align*} 
            By induction, the identity now works for every non-negative integer $n$. 
                
        \end{solution}

        \item Note that $J_{1/2}(\rho) = \sqrt{\dfrac2\pi}\rho^{-1/2}\sin\rho$. 
        
        \begin{solution}
            We simply have
            \begin{align*}
                J_{1/2}(\rho) &= \dfrac{\sqrt{\rho/2}}{\sqrt\pi} \int_{-1}^1 e^{i\rho t} dt\\
                &= \sqrt{\dfrac\rho{2\pi}} \cdot \dfrac{e^{i\rho} - e^{-i\rho}}{i\rho}\\
                &= \sqrt{\dfrac2\pi} \rho^{-1/2}\sin\rho.
            \end{align*}
        \end{solution}

        \item Prove that 
            $$\lim_{n\to-1/2} J_n(\rho) = \sqrt{\dfrac2\pi} \rho^{-1/2} \cos\rho.$$

        \begin{solution}
            \href{https://math.stackexchange.com/questions/4376199/a-definite-integral-in-the-bessel-function-j-1-2s-from-stein-shakarchi/4376415#4376415}{Solution reference here.}

            Directly calculate the limit won't work, so we are looking at a way of raising the power
            of say $(1-t^2)$ from $n-1/2$ to $n+1/2$, which implies $K_{n+1} = K_n/(2n+1)$. 
            
            To avoid typing a lot of constants, we let $K_n=\dfrac1{2^n \Gamma(n+1/2)\sqrt\pi}$. 

            Observing the integrand, we can reflect it: 
            $$J_n(\rho)=2\rho^n K_n\int_0^1 \cos(\rho t) (1-t^2)^{n-1/2}.$$

            Then, we use a slightly non-trivial integration by parts of 
            $u=\dfrac{\cos \rho t}{t}, dv = t(1-t^2)^{n-1/2}$ with
            $du=\dfrac{-t\rho \sin\rho t-\cos\rho t}{t^2}, v = \dfrac{1-(1-t^2)^{n+1/2}}{2n+1}$.

            The additional constant $\dfrac1{2n+1}$ in $v$ is to make the non-integral part easy to calculate,
            which in turns actually make other calculations easier as well.
            Following this setting we have
            \begin{align*}
                J_n(\rho)&= 2\rho^n K_{n+1} \cos\rho + 2\rho^n K_n \int_0^1 (t\rho \sin\rho t + \cos\rho t)\dfrac{1-(1-t^2)^{n+1/2}}{t^2(2n+1)}dt.
            \end{align*}
            We have to show the second term converges to $0$ when $n\to -1/2$, although it should be, I still struggle to find
            ways to prove this, so I am going to leave this here for now.
        \end{solution}

        \item Observe that the formulas we have proved in the text giving $F_0$ in terms of $f_0$ (when describing
        the Fourier transform of a radial function) take the form

        $$F_0(\rho) = 2\pi \rho^{-(d/2)+1} \int_0^\infty J_{(d/2)-1}(2\pi \rho r)f_0(r) r^{d/2}\, dr,$$

        for $d=1,2,$ and $3,$ if one uses the formulas above with the understanding that
        $J_{-1/2}(\rho)=\lim_{n\to-1/2}J_n(\rho)$.
        It turns out that the relation between $F_0$ and $f_0$ given by the equation above is valid in all dimensions $d$.

        \begin{solution}
            I can't solve this bro, it is too hard...
        \end{solution}
    \end{enumerate}

    \item We observed that the solution $u(x,t)$ of the Cauchy problem for the wave equation given by
    formula (3) depends only on the initial data on the base on the backward light cone.
    It is natural to ask if this property is shared by \textit{any} solution of the wave equation.
    An affirmative answer would imply uniqueness of the solution.

    Let $B(x_0, r_0)$ denote the closed ball in the hyperplane $t=0$ centered at $x_0$ and of radius
    $r_0$. The \textbf{backward light cone} with base $B(x_0, r_0)$ is defined by 
    $$\mathcal L_{B(x_0,r_0)} = \{ (x,t)\in\mathbb R^d \times \mathbb R: \abs{x-x_0} \leq r_0-t, \, 0\leq t\leq r_0\}.$$

    \begin{theorem}
        Suppose that $U(x,t)$ is a $C^2$ function on the closed upper half-plane
        $\{(x,t): x\in\mathbb R^d, t\geq 0\}$ that solves the wave equation
        $$\dfrac{\partial^2 u}{\partial t^2} = \Delta u.$$

        If $u(x,0)=\dfrac{\partial u}{\partial t}(x,0)=0$ for all $x\in B(x_0, r_0)$, then
        $u(x,t)=0$ for all $(x,t)\in\mathcal L_{B(x_0,r_0)}$.
    \end{theorem}

    In words, if the initial data of the Cauchy problem for the wave equation vanishes on a ball $B$,
    then \textit{any} solution $u$ of the problem vanishes in the backward light cone with base $B$.
    The following steps outline a proof of the theorem.

    \begin{enumerate}
        \item Assume that $u$ is real. For $0\leq t\leq r_0$ let $B_t(x_0,r_0)=\{x: \abs{x-x_0} \leq r_0-t\}$,
        and also define
        $$\nabla u(x,t) = \left(\dfrac{\partial u}{\partial x_1},\dots,\dfrac{\partial u}{\partial x_d},\dfrac{\partial u}{\partial t}\right).$$

        Now consider the energy integral
        \begin{align*}
            E(t) &= \dfrac12 \int_{B_t(x_0,r_0)}\abs{\nabla u}^2 dx\\
            &= \dfrac12 \int_{B_t(x_0,r_0)} \left(\dpard{u}{t}\right)^2 + \sum_{j=1}^d \left(\dpard{u}{x_j}\right)^2 dx.
        \end{align*}

        Observe that $E(t)\geq 0$ and $E(0)=0$. Prove that
        $$E'(t) = \int_{B_t(x_0, r_0)} \dpard{u}{t} \dpard[2]{u}{t} + \sum_{j=1}^d \dpard{u}{x_j}\dfrac{\partial^2 u}{\partial x_j \partial t}dx - \dfrac12 \int_{\partial B_t(x_0,r_0)} \abs{\nabla u}^2 d\sigma(\gamma).$$

        \begin{solution}
            First I was naively differentiating the integrand and only get the first term, but after that
            realize the domain of integral depends on $t$, which prevents me from doing so! Let's see the proper solution.

            This solution uses \href{https://en.wikipedia.org/wiki/Leibniz_integral_rule}{Leibniz Integral Rule}.
            We can split the integral like so:
            \begin{align*}
                E(t) &= \dfrac12 \int_t^{r_0} \left(\int_{\partial B_r(x_0, r_0)} \abs{\nabla u}^2 d\sigma(\gamma)\right)dr
            \end{align*}
            Note that now only the outer integral depends on $t$, we can thus apply Leibniz integral rule and obtain
            \begin{align*}
                E'(t) &= \dfrac12 \int_t^{r_0} \left(\int_{\partial B_r(x_0, r_0)} \dpard{}{t} \abs{\nabla u}^2 d\sigma(\gamma)\right) dr
                    -\dfrac12 \int_{\partial B_t(x_0, r_0)} \abs{\nabla u}^2 d\sigma(\gamma).
            \end{align*}
            The term related to $r_0$ is differentiated to none as it is only a constant, now it should be obvious the identity is true.
        \end{solution}

        \item Show that 
        $$\dpard{}{x_j}\left[\dpard{u}{x_j} \dpard{u}{t}\right] = \dpard{u}{x_j} \dfrac{\partial^2 u}{\partial x_j\partial t} + \dpard[2]{u}{x_j} \dpard{u}{t}.$$

        \begin{solution}
            This is a standard product rule.
        \end{solution}

        \item Use the last identity, the divergence theorem, and the fact that $u$ solves the wave equation
        to prove that
        $$E'(t) = \int_{\partial B_t(x_0, r_0)} \sum_{j=1}^d \dpard{u}{x_j} \dpard{u}{t} \varv_j d\sigma(\gamma) - \frac12 \int_{\partial B_t(x_0, r_0)} \abs{\nabla u}^2 d\sigma(\gamma),$$
        where $\varv_j$ denotes the $j^{th}$ coordinate of the outward normal to $B_t(x_0, r_0)$. 

        \begin{solution}
            Compare to (a), we only need to show 
            $$\int_{B_t(x_0, r_0)} \dpard{u}{t} \dpard[2]{u}{t} + \sum_{j=1}^d \dpard{u}{x_j}\dfrac{\partial^2 u}{\partial x_j \partial t}dx
            =\int_{\partial B_t(x_0, r_0)} \sum_{j=1}^d \dpard{u}{x_j} \dpard{u}{t} \varv_j d\sigma(\gamma).$$

            With the suggested strategies, we notice the left integrand can be written as
            \begin{align*}
                \dpard{u}{t} \dpard[2]{u}{t} + \sum_{j=1}^d \dpard{u}{x_j}\dfrac{\partial^2 u}{\partial x_j \partial t}
                &= \dpard{u}{t} \dpard[2]{u}{t} - \sum_{j=1}^d \dpard[2]{u}{x_j} \dpard{u}{t} + \sum_{j=1}^d \dpard{}{x_j}\left[\dpard{u}{x_j}\dpard{u}{t}\right]\\
                &= \dpard{u}{t}\left[\dpard[2]{u}{t} - \Delta u\right] + \sum_{j=1}^d \dpard{}{x_j}\left[\dpard{u}{x_j}\dpard{u}{t}\right]\\
                &= \sum_{j=1}^d \dpard{}{x_j}\left[\dpard{u}{x_j}\dpard{u}{t}\right].
            \end{align*}
            Therefore, by letting $F=F_t(x_1, \dots, x_d)$ to be a vector field: 
            $$F_t = \dpard{u}{t} \left\langle \dpard{u}{x_1}, \dots, \dpard{u}{x_d}\right\rangle,$$
            we may apply the Divergence theorem so that
            \begin{align*}
                \int_{B_t(x_0, r_0)} (\nabla \cdot F_t) dx &= \int_{\partial B_t(x_0, r_0)} (F\cdot \overrightarrow{\textbf{n}}) d\sigma(\gamma),
            \end{align*}
            and we have proved the identity.
        \end{solution}
        
        \item Use the Cauchy-Schwarz inequality to conclude that 
        $$\sum_{j=1}^d \dpard{u}{x_j} \dpard{u}{t} \varv_j \leq \dfrac12 \abs{\nabla u}^2,$$

        and as a result, $E'(t)\leq 0$. Deduce from this that $E(t)=0$ and $u=0$.

        \begin{solution}
            Let's write $\overrightarrow{\textbf{n}}=(\varv_1, \dots, \varv_d)$, so that $\sum_{j=1}^d \varv_j^2=1$.
            Then by Cauchy-Schwarz inequality,
            $$\sum_{j=1}^d \left(\dpard{u}{x_j}\right)^2 = \sum_{j=1}^d \left(\dpard{u}{x_j}\right)^2\sum_{j=1}^d \varv_j^2 
            \geq \left( \sum_{j=1}^d\dpard{u}{x_j}\varv_j \right)^2.$$

            Now we adding the partial derivative of $t$ into the terms, giving us
            \begin{align*}
                \dfrac12 \abs{\nabla u}^2 &\geq \frac12\left[ \left( \sum_{j=1}^d\dpard{u}{x_j}\varv_j \right)^2 + \left(\dpard{u}{t}\right)^2 \right]\\
                &\geq \dpard{u}{t} \sum_{j=1}^d \dpard{u}{x_j}\varv_j.
            \end{align*}

            We have proved that $E'(t)\leq 0$ for all $t\in [0, r_0]$.
            To show that the energe $E(t)$ is zero on $[0, r_0]$ and thus subsequently $u=0$ too, we take any $t\in (0,r_0]$,
            then by mean value theorem there is $c\in[0,t]$ such that

            $$\dfrac{E(t)}{t} = \dfrac{E(t)-E(0)}{t} = E'(c)\leq 0 \implies E(t)\leq 0.$$

            Since $E$ is nonnegative, we can conclude $E(t)=0$.

        \end{solution}
    \end{enumerate}

    \addtocounter{enumi}{3}

    \item For every real number $a>0$, define the operator $(-\Delta)^a$ by the formula 
    $$(-\Delta)^a f(x) = \int_{\mathbb R^d} (2\pi\abs\xi)^{2a} \hat f(\xi) e^{2\pi i\xi \cdot x}\, d\xi$$
    whenever $f\in\mathcal S(\mathbb R^d)$.

    \begin{enumerate}
        \item Check that $(-\Delta)^a$ agrees with the usual definition of the $a^{\text{th}}$ power of $-\Delta$ (that is,
        $a$ compositions of minus the Laplacian) when $a$ is a positive integer.

        \begin{solution}
            First we have 
            \begin{align*}
                f(x) &= \int_{\mathbb R^d} \hat f(\xi) e^{2\pi i\xi \cdot x} d\xi\\
                (-\Delta) f(x) &= -\int_{\mathbb R^d} (2\pi i)^2 \abs{\xi}^2 \hat f(\xi) e^{2\pi i\xi \cdot x}d\xi\\
                &= \int_{\mathbb R^d} (2\pi \abs\xi)^2 \hat f(\xi) e^{2\pi i\xi \cdot x}d\xi.
            \end{align*}
            This proves the definition aligns with the usual Laplacian operator at $a=1$. Now assumes it is true for some positive integer $a$,
            then
            \begin{align*}
                (-\Delta) (-\Delta)^a f(x) &= (-\Delta) \int_{\mathbb R^d} (2\pi \abs{\xi})^{2a} \hat f(\xi) e^{2\pi i\xi \cdot x}d\xi\\
                &= -\int_{\mathbb R^d} (2\pi \abs\xi)^{2a} (2\pi i)^2 \abs{\xi}^2 \hat f(\xi) e^{2\pi i\xi \cdot x}d\xi\\
                &= \int_{\mathbb R^d} (2\pi \abs\xi)^{2(a+1)} \hat f(\xi) e^{2\pi i\xi \cdot x}d\xi\\
                &=(-\Delta)^{a+1}f(x),
            \end{align*}
            thus $(-\Delta)^a$ agrees with the $a^{\text{th}}$ power of negative Laplacian for all positive integer $a$ by induction.
        \end{solution}

        \item Verify that $(-\Delta)^a (f)$ is indefinitely differentiable.
        
        \begin{solution}
            For any multi-index $\alpha=(\alpha_1, \dots, \alpha_d)$ where $|\alpha|=\sum_{1\leq i\leq d}\alpha_i$
            (I forgot whether this is the original definition, it probably is),
            we note that 
            $$\left(\dpard{}{x}\right)^\alpha (-\Delta)^a f(x) = \int_{\mathbb R^d}(2\pi \abs\xi)^{2a}\cdot (2\pi i)^{\abs\alpha} \xi^\alpha
                \hat f(\xi) e^{2\pi i\xi\cdot x} \, d\xi$$
            where $\xi^\alpha=\xi_1^{\alpha_1}\xi_2^{\alpha_2}\cdots\xi_d^{\alpha_d}$.
            Since $f\in\mathcal S(\mathbb R^d)$, the integrand is rapidly decreasing, hence the integral value is finite.
        \end{solution}

        \item Prove that if $a$ is not an integer, then in general $(-\Delta)^a (f)$ is not rapidly decreasing.
        \begin{solution}
            
        \end{solution}
    \end{enumerate}
\end{enumerate}



\end{document}
